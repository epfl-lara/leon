\section{Introduction}

\IEEEPARstart{T}{his} template is not trying to suggest a specific
style for the content of your candidacy exam/thesis proposal paper; it
only concerns the style of the writeup itself.  This template is in LaTeX.
We encourage you to use it for your write-up.

If you insist on using a different word processing program please
match the style as much as possible.  Here is the basic format: The
paper size is A4, the font is 10 point, 
single-spacing, ``times roman like.''
Two columns, each 8.75~cm wide, with 1.5~cm left/right margins.
Sections and sub-sections have bold numbered headings.  Figures have
captions underneath, with a bold label and number, and also a short
caption sentence.  Ditto for tables.  The text itself is a mix of standard
paragraphs, bullets, numbered lists, and equations where appropriate.
References are at the end, in a numbered format, in a slightly smaller
font.

The overall length of this write-up should be 4-8 pages. 

\subsection{Purpose of this Write-Up}

This write-up serves two purposes. First, it forms the basis for
your candidacy exam. As such you should summarize the three papers selected
by your advisor and yourself, and analyze as well as discuss them critically.
Second, the write-up 
is also your thesis proposal.  Therefore, the last one or two pages should
be dedicated to your own preliminary work. A road-map of how you plan to
advance the state of the art in your chosen area should also be given.
For further details please consult the document ``PhD Candidacy Exam Overview.'' You can find the latest
version at {\tt http://phd.epfl.ch/page57746-en.html}.


\subsection{What to Put into the Introduction}
Describe briefly the context, the problem, shortcomings in prior
approaches, and your proposed approach and solution. Forecast results.
