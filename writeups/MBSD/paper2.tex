\newcommand{\next}{\bigcirc}
\newcommand{\until}{\mathcal{\; U \;}}
\newcommand{\eventually}{\diamond}
\renewcommand{\eventually}{\diamondsuit}
\newcommand{\always}{\square}

\section{Paper 2: Temporal-Logic-Based Reactive Mission and Monitor Planning \cite{gazit:95}} \label{sec:paper2}

The paper considers the problem of synthesizing transition systems for robotic motion from the 
high-level specifications  provided in linear time temporal logic (LTL). The paper proposes  
(a) a specification language that is a restriction of LTL for specifying the motion of a robot
and also the environment in which it operates.  
(b) a technique for automatically generating a discrete automaton for the robot motion, using the 
existing approaches, that satisfies the specification of the robot when the environment behaves as specified. 
(c) an algorithm for converting the generated discrete automaton to a hybrid automaton that interacts 
continuously with the environment.
(d) a compositional method for generating multirobot tasks using \emph{assume guarantee reasoning}.

\textbf{Specifying the motion of a robot and the environments using GR(1) formula:}

\textit{Robot Model:} Consider a robot is operating in a polygonal space $P$ that 
has several partitions, namely, $P_1,P_2 \cdots P_n$. $p(t)$ is the position of the robot at time $t$
and $u(t)$ is the control input which can be used to move the robot to the desired location. 
$p(t) \in P \subseteq \real^2$ and $u(t) \in \real^2$. 
$\{ r_1,\cdots,r_n \}$ is a set of propositions such that $r_i$ is $true$ iff the robot is in  partition $P_i$.
The robot can perform $k$ actions given by $A$.  For all $1 \le i \le k$, a proposition $a_i$ is true if the
action $i$ is being executed.
Let $Y = \{ r_1,\cdots,r_n, a_1,\cdots,a_k \}$ denote the set of robot propositions.

\textit{Environment Model:}
The robot interacts with the environment through $m$ binary sensors whose values are captured by 
the variables $X = \{ x_i \vbar 1 \le i \le m\}$. 
The behavior of the environment is specified using a LTL formula over sensor the variables.

\textit{Specification Language:} Let $AP = X \cup Y$  denote the set of atomic propositions.
The syntax of the LTL formulas over the set of atomic proposition is given by the following grammar:
\[ \varphi ::= \pi \in AP \vbar \neg \phi \vbar \phi \vee \phi \vbar  \next \phi \vbar \phi \until \phi  \]
where $\next$ represents the next operator and $\until$ the until operator.
Define  the "Eventually" operator $\eventually \phi$ as $true \until \phi$ and the 
always operator $\always \phi$ as $\neg \eventually \neg \phi$.

The paper focus on a special class of LTL formulas called \emph{generalized reactivity formula}(GR(1)) formula
that have the following form: $\phi_e \Rightarrow \phi_s$  where $\phi_e$ is the assumption
about the environment and $\phi_s$ is the desired behavior of the robot.
$\phi_e$ and $\phi_s$ should have the following structure $\phi_i \wedge \phi_t \wedge \phi_g$ where
$\phi_i$ is a nontemporal formula defined over $X$ for $\phi_e$ and over $Y$ for $\phi_s$. 
$\phi_t$ is a temporal formula of the form $\square B_i$ (always $B_i$) where $B_i$ is a formula 
defined over $X \cup Y \cup \next X$ for $\phi_e$ and over $X \cup Y \cup \next X \cup \next Y$ for
$\phi_s$. Finally, $\phi_g$ captures the invariants of the environment and the goal of the robot 
and is of the form $\always \eventually B_i$
where $B_i$ is defined over $X \cup Y$ for both $\phi_e$ and $\phi_s$.

\newcommand{\shoot}{\mathit{a^{shoot}}}
\newcommand{\duck}{\mathit{s^{duck}}}

\textit{Duck Hunting:} The paper discusses an example in which a robot's task is find \emph{Nemo}.
In this report we consider a slight different example in which the robot is stationary but
it has to visually scan through the in the its perspective and look for ducks that may occasionally 
fly through the space. The task of the robot is fire a shot directed towards the duck on detecting one.

Assume that the robot has a 2D vision and the space in front of the robot is divided into a $4 \times 4$ grid.
The robot scans the grid column by column starting from the left bottom. 
The $(i,j)^{th}$ cell of the grid corresponds to a proposition $r_{ij}$.
The robot is fitted with a gun and can direct a shot towards a target in a grid which can be activated 
by the proposition $\shoot$. The robot detects a duck in a grid which a sensor $\duck$.
The robot initially starts at the grid position $(1,1)$ (left bottom).

We first specify the assumptions on the environment $\phi_e = {\phi_i}^e \wedge {\phi_t}^e \wedge {\phi_g}^e$.
${\phi_i}^e = \neg \duck$ i.e, initially there is not duck. 
A duck never stays in a grid it is continuously moving.
${\phi_t}^e = \always ( \bigwedge_{i,j} (r_{ij} \wedge \duck \Rightarrow \next \neg \duck))$.
${\phi_g}^e = \always \eventually true$.

We now specify the behavior of the robot $\phi_s = {\phi_i}^s \wedge {\phi_t}^s \wedge {\phi_g}^s$.
${\phi_i}^s = r_{1,1} \wedge \neg \shoot$ initially the robot scans the grid $(1,1)$ and does not shoot.
${\phi_t}^s$ has three components the first specifies the possible transitions, the second specifies the
mutual exclusion property that the robot scans atmost one square at a time and the final component 
specifies that the robot should shoot when it finds the duck in a grid.
%
\begin{align}
{\phi_t}^s = 
\begin{cases}
\begin{cases}
\always (\bigwedge_{i,j} (r_{ij} \wedge j < 4 \Rightarrow r_{i(j+1)}) \wedge  \\
\qquad (r_{ij} \wedge j=4 \wedge i<4 \Rightarrow r_{(i+1)1}) \wedge  \\
\qquad (r_{ij} \wedge i=4 \wedge j=4 \Rightarrow r_{11}) 
\end{cases} \\

\end{cases}
\end{align}
